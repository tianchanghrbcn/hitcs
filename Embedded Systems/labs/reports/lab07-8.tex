% !TeX program = XeLaTeX 
\documentclass[UTF8,a4paper]{ctexart}

% ---------------- 基础宏包 ----------------
\usepackage{geometry}     % 页面边距
\usepackage{fontspec}     % XeLaTeX 字体
\usepackage{graphicx}     % 插图
\usepackage{pdfpages}     % 直接插入外部 PDF,用于封面
\usepackage{booktabs}     % 表格横线
\usepackage{array}        % 表格列格式
\usepackage{tabularx}     % 自动伸缩列
\usepackage{multirow}     % 表格多行单元格
\usepackage{caption}      % 图表标题
\usepackage{hyperref}     % 超链接
\usepackage{chngcntr}     % 章节内重编号 (图表按章)

% --------------- 字体与排版 ----------------
\setCJKmainfont{SimSun}          % 中文:宋体
\setmainfont{Times New Roman}    % 英文:Times
\setmonofont{Consolas}           % 代码:Consolas

% 正文小四 + 20 pt 行距
\newcommand{\bodyfont}{\zihao{-4}\setlength{\baselineskip}{20pt}}
% 表格内部五号
\newcommand{\tabfont}{\zihao{-5}}

% 页面边距
\geometry{left=2.5cm,right=2.5cm,top=2.5cm,bottom=2.5cm}

% 段前 6 pt,段后 0 pt
\setlength{\parskip}{6pt}
\setlength{\parindent}{2em}

% 各级标题加粗并设字号
\ctexset{
  section/format     = {\bfseries\zihao{3}},
  subsection/format  = {\bfseries\zihao{4}},
  subsubsection/format = {\bfseries\zihao{-4}}
}

% 图表编号随章节重置:表 1-1、图 2-1 ...
\counterwithin{table}{section}
\counterwithin{figure}{section}

% -------------- 文档开始 ------------------
\begin{document}

% 把封面 PDF 的第一页插入为封面
% 将 cover.pdf 改成你的文件名;若只需第一页,用 pages=1
\includepdf[pages=1]{cover.pdf}


\bodyfont
\tableofcontents
\newpage

% ==================================================
\section{应用场景}
\textbf{本节图表索引:} 表~\ref{tab:scenario}。

\begin{table}[htbp]
    \centering
    \caption{Ultra-Green v2.0 场景与目标}
    \label{tab:scenario}
    \tabfont
    \begin{tabularx}{\textwidth}{@{}p{2.5cm}X@{}}
        \toprule
        \textbf{维度} & \textbf{描述} \\ \midrule
        场景 & 连栋薄膜温室,面积约 $8\,\mathrm{m}\times30\,\mathrm{m}$,经济作物(黄瓜/番茄)。 \\
        目标 & 年均温度 $22\pm2\,^\circ\mathrm{C}$、湿度 $60\pm5\,\%$RH;光合有效辐射 $\ge400\,\mu\mathrm{mol\,m^{-2}\,s^{-1}}$;灌溉水节省 $\ge20\%$。 \\
        约束 & 220 VAC + 屋顶光伏;弱网(NB-IoT/LTE Cat-M1);现场维护频次 $\le1$ 次/月。 \\
        \bottomrule
    \end{tabularx}
\end{table}

\vspace{-2em}

% ==================================================
\section{系统整体架构}
\textbf{本节图表索引:} 图~\ref{fig:sys}。整体硬件与网络拓扑如图~\ref{fig:sys} 所示。  
\begin{figure}[htbp]
    \centering
    \includegraphics[width=\textwidth]{system_architecture.pdf}
    \caption{系统组成方框图}
    \label{fig:sys}
\end{figure}

% ==================================================
\section{硬件设计}
\textbf{本节图表索引:} 表~\ref{tab:bom}、表~\ref{tab:inputs}、表~\ref{tab:outputs}。

\subsection{关键器件与 BOM}
\begin{table}[htbp]
    \centering
    \caption{核心器件与参数(千片级)}
    \label{tab:bom}
    \tabfont
    \begin{tabular}{@{}llllp{5cm}@{}}
        \toprule
        \textbf{类别} & \textbf{器件} & \textbf{主要参数} & \textbf{单价} & \textbf{选型理由} \\ \midrule
        MCU & STM32H753ZI & 480 MHz M7/2 MB Flash/1 MB SRAM & \$10 & 充裕 SRAM 支撑 RT-Thread + TLS \& AI Lite \\
        协处理 & STM32L051K8 & 80 µA/MHz;STOP2 1.2 µA & \$2 & 夜间深睡,RTC 唤醒 \\
        Modem & Quectel BG95-M3 & Cat-M1/NB-IoT/GNSS;PSM 5 µA & \$8 & 弱网友好,低功耗 \\
        LoRa SoC & ASR6502 & SX1262 RF + M0 & \$3 & 2 km LOS,Mesh 组网 \\
        eMMC & Kingston 8 GB & −40 ℃–85 ℃ & \$4 & 环形缓冲,双系统 \\
        电源 & TPS54386 + TLV70033 & 3 A Buck + 3.3 V LDO & \$1.5 & 二级供电 \\ \bottomrule
    \end{tabular}
\end{table}

\subsection{原理图 \& PCB 亮点}
\begin{itemize}
    \item \textbf{6-Layer 堆栈}:Signal/GND/5 V/3.3 V/GND/Signal;Ethernet \& RMII 差分 50 Ω。
    \item \textbf{模拟隔离}:CO\textsubscript{2} NDIR \& 土壤 EC 前端独立 AGND。
    \item \textbf{过流\&雷击防护}:24 V TVS + 共模电感;RS485 芯片 SN65HVD78(IEC 61000-4-5 ±15 kV)。
    \item \textbf{可维护性}:USB-C OTG、ST-LINK VCP,全板 0402 封装。
\end{itemize}

\subsection{输入/输出信号}
\begin{table}[htbp]
    \centering
    \caption{主要输入信号}
    \label{tab:inputs}
    \tabfont
    \begin{tabular}{@{}cccccc@{}}
        \toprule
        \# & 类型 & 精度 & 接口 & 速率 & 备注 \\ \midrule
        1 & 温度(SHT45) & ±0.1 ℃ & I²C & 1 Hz & 环境空气 \\
        2 & 湿度(SHT45) & ±1 \%RH & I²C & 1 Hz & 环境空气 \\
        3 & CO\textsubscript{2}(S8) & ±30 ppm & UART & 0.2 Hz & NDIR \\
        4 & 土壤湿度(Cap-EC) & ±2 \% & I²C+ADC & 0.5 Hz & 电容式 \\
        5 & 光照(VEML7700) & ±0.045 lux & I²C & 1 Hz & 可见光 \\
        6 & 风速(Anem) & ±0.3 m/s & RS485 & 1 Hz & 0–30 m/s \\ \bottomrule
    \end{tabular}
\end{table}

\begin{table}[htbp]
    \centering
    \caption{主要输出执行信号}
    \label{tab:outputs}
    \tabfont
    \begin{tabular}{@{}ccp{3cm}p{4cm}@{}}
        \toprule
        \# & 执行对象 & 驱动方式 & 规格 \\ \midrule
        1 & 排风/加热风机 & MOSFET ×4 & 24 V/2 A PWM \\
        2 & 滴灌泵 + 阀门 & 继电器 & 220 VAC/10 A \\
        3 & 补光 LED & 恒流 Buck ×3 & 12 V,0–1 A \\
        4 & 卷膜机/百叶 & CANopen & 24 V 60 W \\
        5 & 蜂鸣 + OLED & GPIO/I²C & 85 dB;128×64 \\ \bottomrule
    \end{tabular}
\end{table}

% ==================================================
\section{软件设计}
\textbf{本节图表索引:} 表~\ref{tab:tasks}、图~\ref{fig:soft}。

\subsection{双核任务栈}
\begin{table}[htbp]
    \centering
    \caption{主要任务与堆栈分配}
    \label{tab:tasks}
    \tabfont
    \begin{tabular}{@{}p{3cm}p{1.8cm}p{1.8cm}p{6cm}@{}}
        \toprule
        处理器/OS & 任务 & 周期/触发 & 职责 \\ \midrule
        \multirow{5}{*}{\shortstack{Cortex-M7\\RT-Thread 4.2}}
          & sense    & 1 s & 传感器采样→环形缓冲 \\
          & ctrl     & 500 ms & PID + 模糊→PWM/Relay \\
          & comm     & 100 ms+IRQ & MQTT(TLS 1.3) 上下行 \\
          & lora\_gw & 20 ms+IRQ & 叶节点 LoRa-MAC 汇聚 \\
          & ui       & 100 ms & 菜单/报警/OTA 进度 \\[2pt]
        \multirow{2}{*}{\shortstack{Cortex-M4\\FreeRTOS}}
          & sleep\_mon & 5 s & 门磁、供电→唤醒 MCU \\
          & pulse\_cnt & IRQ & 电能脉冲计量 \\ \bottomrule
    \end{tabular}
\end{table}

\subsection{软件功能图}
\begin{figure}[htbp]
    \centering
    \includegraphics[width=\textwidth]{software_flow.pdf}
    \caption{软件功能方框图}
    \label{fig:soft}
\end{figure}

% ==================================================
\section{开发工具与里程碑}
\textbf{本节图表索引:} 表~\ref{tab:tools}、表~\ref{tab:mile}。

\begin{table}[htbp]
    \centering
    \caption{主要开发工具}
    \label{tab:tools}
    \tabfont
    \begin{tabular}{@{}p{3cm}p{4cm}p{6cm}@{}}
        \toprule
        流程块 & 工具 & 说明 \\ \midrule
        MCU 固件 & VS Code + GNU ARM GCC 12 & C/C++17;CMake;OpenOCD/J-Link;RT-Thread Env \\
        LoRa 协议 & SX1262 DevKit + Semtech Studio & FSK/GFSK;Mesh 调试 \\
        AI 控制 & TensorFlow-Lite-Micro & LSTM 温度预测(M7 FPU) \\
        云端 & Docker Compose & EMQX 5 + TimescaleDB + Grafana 10 \\
        手机端 & uni-app + Vant & 实时曲线 + 推送告警 \\
        PCB & KiCad 8/Altium 24 & ERC、DRC、3D \\
        机械结构 & SolidWorks 2023 & IP55 外壳 + 散热风道 \\ \bottomrule
    \end{tabular}
\end{table}
\vspace{-3em}
\begin{table}[htbp]
    \centering
    \caption{项目进度规划}
    \label{tab:mile}
    \tabfont
    \begin{tabular}{@{}cp{6cm}c@{}}
        \toprule
        阶段 & 核心输出 & 典型耗时 \\ \midrule
        1 & SRS v1.0、功能分配表 & 1 周 \\
        2 & SCH、Gerber、BOM & 3 周 \\
        3 & RT-Thread BSP 跑灯 & 1 周 \\
        4 & HAL 单元测试 & 2 周 \\
        5 & PID/模糊调参 & 1 周 \\
        6 & MQTT 贯通、Grafana 面板 & 1 周 \\
        7 & LoRa Mesh ≥95\% & 1 周 \\
        8 & A/B rollback;TLS 测试 & 1 周 \\
        9 & 72 h 烘箱 + 功耗 & 2 周 \\
        10 & Demo 视频 & 1 周 \\ \bottomrule
    \end{tabular}
\end{table}

% ==================================================
\section{验收与评测}
\textbf{本节图表索引:} 表~\ref{tab:evaluate}。

\begin{table}[htbp]
    \centering
    \caption{关键评测指标}
    \label{tab:evaluate}
    \tabfont
    \begin{tabular}{@{}llll@{}}
        \toprule
        类别 & 指标 & 要求 & 测试方法 \\ \midrule
        功能 & 温度稳定度 & ±1.0 ℃ & 恒温箱扰动 \\
        功能 & 湿度稳定度 & ±3 \%RH & 湿度腔体阶跃 \\
        通信 & 云端丢包率 & <1 \%/h & EMQX 统计 \\
        通信 & LoRa 覆盖 & ≥2 km & 实地测速 \\
        能耗 & 平均功耗 & <2.5 W & 台式功率计 \\
        异常 & 断网自控 & >24 h & LTE 断链 \\
        安全 & TLS 握手成功率 & 100\% & \texttt{sslyze} 扫描 \\ \bottomrule
    \end{tabular}
\end{table}

% ==================================================
\section{结果预测}
\textbf{本节图表索引:} 表~\ref{tab:result}。

\begin{table}[htbp]
    \centering
    \caption{环境控制精度预测}
    \label{tab:result}
    \tabfont
    \begin{tabular}{@{}lcc@{}}
        \toprule
        测试项 & 目标值 & 预测 RMS 误差 \\ \midrule
        温度 & 22 ℃ & ±0.92 ℃ \\
        湿度 & 60 \%RH & ±2.7 \%RH \\
        土壤含水率 & 30 \% & ±1.8 \% \\ \bottomrule
    \end{tabular}
\end{table}

% ==================================================
\section{结论}
本报告详细阐述了 Ultra-Green v2.0 智能温室平台的硬件设计、软件架构、BOM、开发里程碑及评测指标,满足“≥4 输入、≥2 输出、硬软深度融合、具体应用场景”等要求,为后续视觉识别与光伏-储能协同调度奠定基础。

% ---------------- 参考文献 -----------------
\begin{thebibliography}{9}
\bibitem{stm32h7} STMicroelectronics.\ \emph{STM32H7 Series Reference Manual}, 2024.
\bibitem{rtt} RT-Thread Community.\ \emph{RT-Thread Programmer’s Guide}, 2025.
\bibitem{bg95} Quectel.\ \emph{BG95-M3 Hardware Design}, 2024.
\bibitem{asr6502} ASR Microelectronics.\ \emph{ASR650x Datasheet}, 2023.
\end{thebibliography}

\end{document}